\documentclass[12pt]{report}
\usepackage[utf8]{inputenc}
\usepackage[margin=1in]{geometry} 
\usepackage{amsmath,amsthm,amssymb}
\usepackage{mathrsfs}
\usepackage{cite}
\usepackage{algorithm2e}

\usepackage{enumitem}
%\usepackage[shortlabels]{enumerate}
\usepackage{tikz-cd}
\usepackage{mathtools}
\usepackage{amsfonts}
\usepackage{amscd}
\usepackage{makeidx}
\usepackage{enumitem}
\title{Algorithms for generalized Kac-Moody algebras}
\author{Aidan Backus}
\date{December 2019}

\newcommand{\NN}{\mathbb{N}}
\newcommand{\ZZ}{\mathbb{Z}}
\newcommand{\QQ}{\mathbb{Q}}
\newcommand{\RR}{\mathbb{R}}
\newcommand{\CC}{\mathbb{C}}
\newcommand{\CP}{\mathbb{CP}}
\newcommand{\DD}{\mathbb{D}}

\newcommand{\AAA}{\mathcal A}
\newcommand{\BB}{\mathcal B}
\newcommand{\HH}{\mathcal H}

\newcommand{\AssAlg}{\mathbf{AssAlg}}
\newcommand{\Vect}{\mathbf{Vect}}
\newcommand{\Grp}{\mathbf{Grp}}
\newcommand{\LieAlg}{\mathbf{LieAlg}}
\newcommand{\Mod}{\mathbf{Mod}}
\newcommand{\Rep}{\mathbf{Rep}}
\newcommand{\Set}{\mathbf{Set}}

\DeclareMathOperator*{\ad}{ad}
\newcommand{\Aut}{\operatorname{Aut}}
\newcommand{\Cantor}{\mathcal{C}}
\DeclareMathOperator*{\crk}{corank}
\newcommand{\D}{\mathcal{D}}
\newcommand{\card}{\operatorname{card}}
\DeclareMathOperator*{\coker}{coker}
\newcommand{\diam}{\operatorname{diam}}
\newcommand{\End}{\operatorname{End}}
\DeclareMathOperator*{\esssup}{ess\,sup}
\newcommand{\FF}{\mathcal{F}}
\newcommand{\GL}{\operatorname{GL}}
\newcommand{\Hom}{\operatorname{Hom}}
\newcommand{\id}{\operatorname{id}}
\newcommand{\Ind}{\operatorname{Ind}}
\newcommand{\interior}{\operatorname{int}}
\newcommand{\lcm}{\operatorname{lcm}}
\newcommand{\Lip}{\operatorname{Lip}}
\newcommand{\MM}{\mathcal M}
\newcommand{\OO}{\mathcal{O}}
\newcommand{\PGL}{\operatorname{PGL}}
\newcommand{\pic}{\vspace{30mm}}
\newcommand{\pset}{\mathcal{P}}
\DeclareMathOperator*{\rk}{rank}
\newcommand{\Res}{\operatorname{Res}}
\newcommand{\Riem}{\mathcal{R}}
\newcommand{\Sch}{\mathcal{S}}
\newcommand{\SL}{\operatorname{SL}}
\newcommand{\spn}{\operatorname{span}}
\newcommand{\supp}{\operatorname{supp}}

\newcommand{\altrep}{\rho_{\text{alt}}}
\newcommand{\trivrep}{\rho_{\text{triv}}}
\newcommand{\regrep}{\rho_{\text{reg}}}
\newcommand{\stdrep}{\rho_{\text{std}}}

\newcommand{\g}{\mathfrak g}
\newcommand{\h}{\mathfrak h}
\newcommand{\gl}{\mathfrak {gl}}
\renewcommand{\sl}{\mathfrak {sl}}

\def\Xint#1{\mathchoice
{\XXint\displaystyle\textstyle{#1}}%
{\XXint\textstyle\scriptstyle{#1}}%
{\XXint\scriptstyle\scriptscriptstyle{#1}}%
{\XXint\scriptscriptstyle\scriptscriptstyle{#1}}%
\!\int}
\def\XXint#1#2#3{{\setbox0=\hbox{$#1{#2#3}{\int}$ }
\vcenter{\hbox{$#2#3$ }}\kern-.6\wd0}}
\def\ddashint{\Xint=}
\def\dashint{\Xint-}

\renewcommand{\Re}{\operatorname{Re}}
\renewcommand{\Im}{\operatorname{Im}}

\newcommand{\dfn}[1]{\emph{#1}\index{#1}}
 	
\newtheorem{theorem}{Theorem}[chapter]
\newtheorem{prop}[theorem]{Proposition}
\newtheorem{lemma}[theorem]{Lemma}
\newtheorem{proposition}[theorem]{Proposition}
\newtheorem{corollary}[theorem]{Corollary}
\theoremstyle{definition}
\newtheorem{definition}[theorem]{Definition}
\newtheorem{remark}[theorem]{Remark}
\newtheorem{example}[theorem]{Example}
\newtheorem{conjecture}[theorem]{Conjecture}
\newtheorem{exercise}[theorem]{Exercise}
\newtheorem{problem}[theorem]{Problem}

\begin{document}
\chapter{blah}
\section{Notation}
Fix a symmetrizable Cartan matrix $A = (a_{ij})_{i,j=1}^d$. We decompose $A = DB$ where $D$ is a diagonal matrix with entries $\varepsilon_1, \dots, \varepsilon_d$ and $B = (b_{ij})_{ij}$ is symmetric. Then $A$ determines a unique Kac-Moody algebra $\g$, equipped with a Cartan subalgebra $\h \subseteq \g$ and an indexed set $\Delta_{simp} = \{\alpha_1, \dots, \alpha_d\} \subset \h^*$ of simple roots. We will always take $\Delta_{simp}$ as the basis of its span, so when we take dot products $\cdot$, they are with $\Delta_{simp}$ as an orthonormal basis.

Let $Q$ denote the root lattice of $\g$; i.e. the lattice in $\h^*$ generated by $\Delta_{simp}$.
\begin{definition}
	The \dfn{multiplicity} $m(\beta)$ of $\beta \in Q$ is the dimension of the vector space $\g_\beta$ of $g \in \g$ such that for every $h \in \h$,
	$$[h, g] = \beta(h)(g).$$
	If $m(\beta) > 0$, we say that $\Delta$ is a \dfn{root} of $\g$ and write $\beta \in \Delta$.
\end{definition}
If $\beta \in \Delta$, then either all coordinates of $\beta$ are positive (i.e. $\geq 0$) or they are all negative. The set of positive $\beta$ is called $\Delta^+$. One has $m(\beta) = m(-\beta)$, so for the purposes of computing root multiplicities, one might as well assume $\beta \in \Delta^+$ (and henceforth we do).

Let $(\cdot,\cdot)$ denote the Killing form of $\g$, so $(\beta, \gamma) = \beta \cdot B\gamma$.
\begin{definition}
	Let $\beta \in \Delta^+$. If $(\beta, \beta) > 0$, then we say that $\beta$ is \dfn{real} and write $\beta \in \Delta^r$. Otherwise, we say that $\beta$ is \dfn{imaginary} and write $\beta \in \Delta^i$.
\end{definition}
\begin{definition}
	The \dfn{fundamental reflection} of $\alpha_i \in \Delta_{simp}$ is
	$$w_i(\beta) = \beta - (\beta, \alpha_i)\alpha_i.$$
	The group generated by the fundamental reflections is the \dfn{Weyl group} $W$.
\end{definition}
	So $\Delta^r$ is the closure of $\Delta_{simp}$ under fundamental reflections. 
\begin{definition}
	The \dfn{imaginary fundamental chamber} $\Delta_{fun}$ is the set of all positive imaginary roots $\beta \in \Delta^i$ such that for every $\alpha_j \in \Delta_{simp}$, $(\beta, \alpha_j) > 0$.
\end{definition}
For any $\beta \in Q$, we let $|\beta|$ denote its height, i.e. the sum of its coordinates with respect to $\Delta_{simp}$. So if $\rho$ is such that for every $\alpha_i \in \Delta_{simp}$, $(2\rho, \alpha_i) = (\alpha_i, \alpha_i)$, then we have $2(\rho, \beta) = |\beta|$.

\begin{definition}
	A \dfn{divisor} of $\beta$ is a $\gamma \in Q^+$ such that there is a $n \in \NN$ satisfying $n\gamma = \beta$. In this case, we write $\gamma|\beta$.
\end{definition}
With this definition in mind, we define
$$c(\beta) = \sum_{\gamma_d|\beta} \frac{m(\gamma_d)}{d}$$
where we have $d\gamma_d = \beta$. This sum appears in the Peterson recurrence formula and is the key to computing root multiplicities.
We define $\gcd \beta$ to denote the $\gcd$ of the coordinates of $\beta$ (with respect to $\Delta_{simp}$).

\section{Generating the set of roots}
We can use the action of the Weyl group to compute $\Delta^r$ from $\Delta_{simp}$. More specifically, we use the following \dfn{pingpong algorithm}:

\begin{algorithm}[H]
	\KwData{a root $\alpha \in \Delta^+$, a maximum height $h$}
	$\ell := \text{Stack}(\alpha)$\;
	\While{$\ell \neq \emptyset$}{
		$\beta := \text{Pop}(\ell)$\;
		$P := \{w_1(\beta), \dots, w_d(\beta)\}$\;
		$P := \{\gamma \in P: |\gamma| \leq h \text{ and } \gamma \geq 0\}$\;
		\For{$\gamma \in P \setminus \Delta^+$}{
			$\Delta^+ := \Delta^+ \cup \gamma$\;
			Push$(\ell, \gamma)$\;
			$m(\gamma) := m(\alpha)$\;
			$c(\gamma) := c(\alpha)$\;
		}
	}
\end{algorithm}

The pingpong algorithm will add $w\alpha$ to $\Delta^+$ for every $w \in W$ such that $|w\alpha| \leq h$, along with recording the values of $m(\gamma)$ and $c(\gamma)$ for $\gamma$ in the orbit, which are preserved by the action of the Weyl group. Indeed, let $w = w_{i_1} \dots w_{i_k}$ and assume that $w^\flat = w_{i_2} \dots w_{i_k}$ is such that $w^\flat \alpha$ been added to $\Delta^+$. Then $w\alpha = w_{i_1}w^\flat \alpha$ and so $w\alpha \in P$. Therefore the claim follows by induction.

After initializing each of the $m(\alpha_j) = c(\alpha_j) = 1$, and pingponging each of the $\alpha_j \in \Delta_{simp}$, we have generated all of $\Delta^r$ up to height $h$. We now must generate the imaginary roots $\Delta^i$. Similar to the case of real roots, we simply must choose one root from each orbit, and to this end we compute the imaginary roots from the imaginary fundamental chamber, $\Delta_{fun}$.

We now recall the following theorem of convex geometry, proven for example in Bruns-Gubeladze \cite{bruns2009polytopes}.
\begin{theorem}[Gordan]
	Let $\Gamma$ be a rational convex polyhedral cone in $\RR^d$ with dual cone
	$$\Gamma^* = \{y \in \RR^d: \forall x \in \Gamma ~y\cdot x \geq 0\}.$$
	If $G = (G, +)$ is the semigroup of lattice points in $\Gamma^*$, then $G$ is finitely generated.
\end{theorem}
\begin{corollary}
	$\Delta_{fun}$ is contained in a semigroup which admits a \dfn{Hilbert basis}; i.e. a minimal, finite generating set.
\end{corollary}
\begin{proof}
	Let $\Gamma \subset \h$ be the fundamental chamber of coroots of $\g$. Then $\Gamma$ is defined by the inequality $Bx \geq 0$, $\Gamma$ is polyhedral, and rational since the entries of $\g$ are integers. Now $\Delta_{fun}$ is contained in the semigroup of lattice points $G$ of the dual cone of $\Gamma$ \cite[\S 5.8]{kac_2014}. By Gordan's theorem, $G$ is finitely generated, so we take as our Hilbert basis a generating set of minimal cardinality.
\end{proof}

The Hilbert basis $\beta_1, \dots, \beta_k$ of $\Gamma^*$ can be computed efficiently from the Cartan matrix of $\g$ by e.g. the Elliot-MacMahon algorithm \cite{pasechnik2001computing}. In our implementation we use polymake \cite{polymake2000}'s implementation of the Elliot-MacMahon algorithm. From the Hilbert basis, any $\beta \in \Delta_{fun}$ can be written uniquely as a linear combination of the $\beta_j$.

The advantage of this approach is that naively iterating over all elements of the root lattice of height at most $h$, for a $d$-dimensional Cartan matrix, is the same as iterating over the set of all different ways to put $h$ balls in $d + 1$ boxes (the last box for when the vector's entries add up to strictly less than $h$), which is a set of cardinality $\binom{n-1}d$. But the number of pseudoroots is in general much lower than $\binom{n-1}d$, especially if $\g$ is affine or even just hyperbolic. For example, to compute the root multiplicities of the hyperbolic Lie algebra $E_{10}$ up to height $60$, we must iterate over $3 \cdot 10^3$ real roots. Meanwhile $\binom{59}{10} \approx 6 \cdot 10^{10}$. On the other hand, the set of real roots is exponential in cardinality. (TODO: Somehow optimize on this!)


\section{Computing root multiplicities}
	We are almost ready to give the algorithm that computes root multiplicities, but to do so, first we must state some preliminary results.
\begin{definition}
	Let $\beta \in \Delta_{fun}$. If $\gamma \in Q^+$ is such that $\beta - \gamma \in Q^+$, then we say that $\gamma$ is \dfn{under} $\beta$, and write $\gamma \prec \beta$.
\end{definition}
	It is immediate that $\preceq$ is a partial order, and that if $\gamma \prec \beta$, then $|\gamma| < |\beta|$.

	For the remainder of this section, fix a $\beta \in \Delta_{fun}$. We must compute $c(\beta)$. By pingponging $\beta$, we immediately have $c(w\beta)$ for every $w \in W$ with $|w\beta| \leq h$. If $w \neq 1$, then $\beta \prec w\beta$. We make the inductive assumption that if $\gamma \prec \beta$ and $(\gamma, \gamma) \leq 0$, then we know $c(\gamma)$. This can be guaranteed by ordering $\Delta_{fun}$ by height. Indeed, if $\gamma \prec \beta$, there is a $w \in W$ such that $w\gamma \in \Delta_{fun}$, and $w\gamma \preceq \gamma \prec \beta$ and so $|w\gamma| < |\beta|$, so we have already computed $c(w\gamma)$ and so $c(\gamma)$ before arriving at $\beta$ in the iteration of $\Delta_{fun}$. So, in particular, we can compute
	$$m(\beta) = c(\beta) - \sum_{\gamma_d|\beta} \frac{m(\gamma_d)}{d}$$
	where $d\gamma_d = \beta$.

To compute $c(\beta)$, we use Peterson's recurrence formula.
\begin{theorem}[Peterson's recurrence formula]
	One has
	$$c(\beta) = \frac{1}{( \beta, \beta - 2\rho)}\sum_{\gamma \prec \beta} ( \gamma, \beta - \gamma ) c(\gamma) c(\beta - \gamma).$$
\end{theorem}
Peterson's recurrence formula is proven, for example, in Kac's book \cite{kac_2014}. Notice that $\gamma \prec \beta$ if and only if $\beta - \gamma \prec \beta$. So we have
$$c(\beta) = \frac{2}{( \beta, \beta) - |\beta|}\sum_{\substack{\gamma \prec \beta\\\gamma \ll \beta - \gamma}} ( \gamma, \beta - \gamma ) c(\gamma) c(\beta - \gamma)$$
where $\ll$ is any total ordering on $Q^+$.

Recall that if $\alpha \in \Delta^r$, then for no $n \geq 2$ is $n \alpha$ a root. So we need to compute $c(n\alpha)$ for every $\alpha$ and every $n$ such that $n \alpha \prec \beta$. To do this, we need a lemma.
\begin{lemma}
	Let $\ell = \gcd \gamma$ and assume $(\gamma, \gamma) > 0$. If $\gamma/\ell \in \Delta$, then $c(\gamma) = 1/\ell$. Otherwise, $c(\gamma) = 0$.
\end{lemma}
\begin{proof}
	We write $\gamma = \sum_i \gamma^i \alpha_i$. We first claim that if $w \in W$ and $\ell = 1$, then $\gcd w\gamma = 1$. Indeed, one has
	$$w_j\gamma = \sum_i \gamma^i (\alpha_i - a_{ij} \alpha_j) = \sum_{i \neq j} \gamma^i \alpha_i + (\gamma^j - \sum_i a_{ij} \gamma^i) \alpha_j$$
	and
	$$\gamma \sum_{i\neq j} (d_i + d_j a_{ij}) \gamma^i + d_j \left(\gamma^j - \sum_i a_{ij} \gamma^i\right) = \sum_i d_i \gamma^i = \ell.$$
	Therefore Bezout's theorem implies that $\gcd w\gamma = 1$. From this it follows that if $\gamma \in \Delta$, then $\ell = 1$.

	Since $(\gamma, \gamma) > 0$, there is at most one $\gamma^\flat \in Q$ in the span of $\gamma$ such that $\gamma \in \Delta$. If $\gamma \in \Delta$, then the above argument shows that $c(\gamma) = m(\gamma) = 1$ and $\ell = 1$. Otherwise, since $c(\gamma) > 0$ and $(\gamma, \gamma) > 0$, there is a $\gamma^\flat|\gamma$ with $m(\gamma^\flat) = 1$. Since we then have $\gcd \gamma^\flat = 1$, it follows that $\ell \gamma^\flat = \gamma$, so the claim follows from definition of $c$.
\end{proof}
	% TODO: Cite the Bezout's thing.
Let $\gamma \prec \beta$. It follows from our inductive assumption that if we do not already know the value of $c(\gamma)$, then $(\gamma, \gamma) > 0$, so the lemma gives an easily computed formula for $c(\gamma)$.












\section{Verification}
We compared our root multiplicities to the multiplicities of $E_{10}$ and $E_{11}$ given by Kleinschmidt \cite{kleinschmidt2004e11} and the multiplicities of $H_2$ and $H_3$ given by Kac \cite{kac_2014}.


\section{Conjectures}
Let $\beta$ be an imaginary root of $\g$ with height $|\beta|$. One has the trivial estimate
$$m(\beta) \leq d^{|\beta|}$$
and we conjecture that this estimate is sharp in the sense that if $\g$ is \emph{any} Kac-Moody algebra of rank $d$ with a nontrivial Tits cone and connected Dynkin diagram, then there is a root $\beta$ such that $m(\beta) = d^{|\beta|}$ (or in the sense that there is a sequence of roots $\beta_n$ such that $m(\beta_n) \sim d^{|\beta_n|}$?) (TODO: Prove it.)

Moreover, we conjecture that multiplicities grow as
$$m(\beta) \approx Cd^{\sqrt{-(\beta, \beta)}}.$$



\bibliography{algorithm}{}
\bibliographystyle{plain}

\end{document}

